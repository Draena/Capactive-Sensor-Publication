\documentclass[11pt,twoside,a4paper]{article}
\usepackage{latexsym}
\newcommand*\rfrac[2]{{}^{#1}\!/_{#2}}

\begin{document}
	\title{Carbon Nanotube Sensors for Synthetic Skin}
	\author{Hamish Colenso}
	\date{December 2015}
	\maketitle
	
	\begin{abstract}
		Hello World.
	
	\end{abstract}
	
	\newpage
	\section{Introduction}
		\subsection{Orthotic Rehabilitation}
		\subsection{Carbon Nanotubes}
		\subsection{something something something}
	\newpage
	\section{Method}
		\subsection{Preparation}
		\subsection{Evaporation}
		\subsection{CNT Deposition}
		\subsection{Optional ILL}
		\subsection{Optional RIE}
		\subsection{Dielectric Layer}
		\subsection{Device Encapsulation}
	\newpage
	\section{Results}
		\begin{enumerate}
			\item Capacitive sensors for strain and touch applications \newline
				This should show that it is possible to both compress the sensor, that is change the dielectric thickness for a response. It also should show that it is possible to stretch 				the sensor, both compressing the dielectric thickness and decreasing the plate coverage area. The devices should be able to withstand 200\% strain and decent 					compressive forces.
			\item $ \rfrac{\Omega}{\Box} $ film characterisation \newline
				This should show that the gauge factor increases as we approach sensor destruction during stretch events. So we have a trade off between a sensor that is reliable 	
				and a sensor that provides the performance characteristics required for mobile applications.
			\item Overshoot removal and power improvements \newline
				This should show that the capacitive sensors remove overshoot when compared to resistive sensors, and that the dielectric leakage from the devices is much less					than the resistive sensors. This will indicate the sensors have improved power consumption performance for mobile applications when compared to the resistive 						devices
		\end{enumerate}
	\newpage
	\section{Evaluation}
		\subsection{Comparison to objectives for orthotic rehabilitation}
	\newpage
	\section{Conclusion}
		\subsection{Indicative results indicate the potentional for this to be applied to rehabilitation devices}
\end{document}